\documentclass[]{article}
\usepackage[T1]{fontenc}
\usepackage{lmodern}
\usepackage{amssymb,amsmath}
\usepackage{ifxetex,ifluatex}
\usepackage{fixltx2e} % provides \textsubscript
% use upquote if available, for straight quotes in verbatim environments
\IfFileExists{upquote.sty}{\usepackage{upquote}}{}
\ifnum 0\ifxetex 1\fi\ifluatex 1\fi=0 % if pdftex
  \usepackage[utf8]{inputenc}
\else % if luatex or xelatex
  \ifxetex
    \usepackage{mathspec}
    \usepackage{xltxtra,xunicode}
  \else
    \usepackage{fontspec}
  \fi
  \defaultfontfeatures{Mapping=tex-text,Scale=MatchLowercase}
  \newcommand{\euro}{€}
\fi
% use microtype if available
\IfFileExists{microtype.sty}{\usepackage{microtype}}{}
\usepackage[margin=1in]{geometry}
\usepackage{color}
\usepackage{fancyvrb}
\newcommand{\VerbBar}{|}
\newcommand{\VERB}{\Verb[commandchars=\\\{\}]}
\DefineVerbatimEnvironment{Highlighting}{Verbatim}{commandchars=\\\{\}}
% Add ',fontsize=\small' for more characters per line
\usepackage{framed}
\definecolor{shadecolor}{RGB}{248,248,248}
\newenvironment{Shaded}{\begin{snugshade}}{\end{snugshade}}
\newcommand{\KeywordTok}[1]{\textcolor[rgb]{0.13,0.29,0.53}{\textbf{{#1}}}}
\newcommand{\DataTypeTok}[1]{\textcolor[rgb]{0.13,0.29,0.53}{{#1}}}
\newcommand{\DecValTok}[1]{\textcolor[rgb]{0.00,0.00,0.81}{{#1}}}
\newcommand{\BaseNTok}[1]{\textcolor[rgb]{0.00,0.00,0.81}{{#1}}}
\newcommand{\FloatTok}[1]{\textcolor[rgb]{0.00,0.00,0.81}{{#1}}}
\newcommand{\CharTok}[1]{\textcolor[rgb]{0.31,0.60,0.02}{{#1}}}
\newcommand{\StringTok}[1]{\textcolor[rgb]{0.31,0.60,0.02}{{#1}}}
\newcommand{\CommentTok}[1]{\textcolor[rgb]{0.56,0.35,0.01}{\textit{{#1}}}}
\newcommand{\OtherTok}[1]{\textcolor[rgb]{0.56,0.35,0.01}{{#1}}}
\newcommand{\AlertTok}[1]{\textcolor[rgb]{0.94,0.16,0.16}{{#1}}}
\newcommand{\FunctionTok}[1]{\textcolor[rgb]{0.00,0.00,0.00}{{#1}}}
\newcommand{\RegionMarkerTok}[1]{{#1}}
\newcommand{\ErrorTok}[1]{\textbf{{#1}}}
\newcommand{\NormalTok}[1]{{#1}}
\ifxetex
  \usepackage[setpagesize=false, % page size defined by xetex
              unicode=false, % unicode breaks when used with xetex
              xetex]{hyperref}
\else
  \usepackage[unicode=true]{hyperref}
\fi
\hypersetup{breaklinks=true,
            bookmarks=true,
            pdfauthor={},
            pdftitle={OLS forecast with BMA lag length selection},
            colorlinks=true,
            citecolor=blue,
            urlcolor=blue,
            linkcolor=magenta,
            pdfborder={0 0 0}}
\urlstyle{same}  % don't use monospace font for urls
\setlength{\parindent}{0pt}
\setlength{\parskip}{6pt plus 2pt minus 1pt}
\setlength{\emergencystretch}{3em}  % prevent overfull lines
\setcounter{secnumdepth}{0}

\title{OLS forecast with BMA lag length selection}
\author{}
\date{}

\begin{document}

\begin{center}
\huge OLS forecast with BMA lag length selection \\[0.2cm]
\normalsize
\end{center}


In order to compute a forecast the optimal lag length connecting the
past and the present has to be estimated. Frequently, the lag length is
selected starting with a lag of 1 up to a optimal lag length. Optimality
is decided upon using one of the different information criterion in
order to weigh the cost of estimation accuracy loss against potential
gains of using more information in form of more lags. However, thereby
all intermediate lags between 1 and the optimal length are automatically
included regardless of there benefit. This reduces the computational
effort considerably: If all possible combinations of lags were estimated
in order to find the best fit, this would be very time intensive. E.g.,
instead of estimating a model with 1 lag, 2 lags or 3 lags and selecting
the model with the smallest BIC, addittionally models with lag 1 and 3,
lag 2 and 3, lag 3, and lag 2, or simply regressing on a constant would
have to be estimated, as well.

However, using the Bayesian model averaging (BMA) package that
implements a grid search is a nice alternative. Here all possible
combinations of the independent variables (in the forecast context this
would be lags of the endogenous and lagged values of other explanatory
variables) are searched for the best combination. The best model is then
the one that has the highest posterior likelihood. Assuming there is no
prior knowledge about the usefullness of any specific lag, an
uninformative prior is selected.

\begin{Shaded}
\begin{Highlighting}[]
\NormalTok{lag.exact<-function(x.df,lag.length)\{}
        \CommentTok{# Returns a dataframe of the lags of x.df with lag.length}
        \NormalTok{x.n=}\KeywordTok{ncol}\NormalTok{(x.df)}
        \NormalTok{x.obs=}\KeywordTok{nrow}\NormalTok{(x.df)}
        \NormalTok{x.df.lag=}\KeywordTok{data.frame}\NormalTok{(}\KeywordTok{matrix}\NormalTok{(}\OtherTok{NA}\NormalTok{,}\KeywordTok{nrow}\NormalTok{(x.df),x.n))}
        \NormalTok{x.names=}\KeywordTok{colnames}\NormalTok{(x.df)}
        \NormalTok{x.df.lag[(lag.length}\DecValTok{+1}\NormalTok{):}\KeywordTok{nrow}\NormalTok{(x.df),]=x.df[}\DecValTok{1}\NormalTok{:(}\KeywordTok{nrow}\NormalTok{(x.df)-lag.length),]}
        \KeywordTok{rownames}\NormalTok{(x.df.lag)=}\KeywordTok{rownames}\NormalTok{(x.df)}
        \KeywordTok{colnames}\NormalTok{(x.df.lag)=}\KeywordTok{paste}\NormalTok{(}\KeywordTok{colnames}\NormalTok{(x.df),}\StringTok{'L'}\NormalTok{,lag.length,}\DataTypeTok{sep=}\StringTok{''}\NormalTok{)}
        \KeywordTok{return}\NormalTok{(x.df.lag)}
        \NormalTok{\}}

\CommentTok{# saving function to disk}
\KeywordTok{dump}\NormalTok{(}\StringTok{'lag.exact'}\NormalTok{,}\KeywordTok{paste}\NormalTok{(DirCode,}\StringTok{'/lag.exact.R'}\NormalTok{,}\DataTypeTok{sep=}\StringTok{''}\NormalTok{),}\DataTypeTok{control=}\StringTok{'all'}\NormalTok{)}
\end{Highlighting}
\end{Shaded}

Next, a function returning a dataframe fo the lags starting with lag
length = horizon and ending with lag length = horizon+max.lag is
computed and saved to file.

\begin{Shaded}
\begin{Highlighting}[]
\NormalTok{lag.hormax=function(df,horizon,maxlag)\{}
        \CommentTok{# returns a lag matrix of the lags of dataframe df from lag=horizon to lag=maxlag.}
        \NormalTok{x.n=}\KeywordTok{ncol}\NormalTok{(df)}
        \NormalTok{x.obs=}\KeywordTok{nrow}\NormalTok{(df)}
        \NormalTok{for (lag in horizon:(horizon+maxlag}\DecValTok{-1}\NormalTok{))\{}
                \NormalTok{if (lag==horizon)\{df.hormax=}\KeywordTok{lag.exact}\NormalTok{(df,horizon)\}else\{}
                        \NormalTok{df.hormax=}\KeywordTok{cbind}\NormalTok{(df.hormax,}\KeywordTok{lag.exact}\NormalTok{(df,lag))}
                        \NormalTok{\}}
                \NormalTok{\}}
        \KeywordTok{return}\NormalTok{(df.hormax)}
        \NormalTok{\}}
\KeywordTok{dump}\NormalTok{(}\StringTok{'lag.hormax'}\NormalTok{,}\KeywordTok{paste}\NormalTok{(DirCode,}\StringTok{'/lag.hormax.R'}\NormalTok{,}\DataTypeTok{sep=}\StringTok{''}\NormalTok{),}\DataTypeTok{control=}\StringTok{'all'}\NormalTok{)}
\end{Highlighting}
\end{Shaded}

Now, a function to select the most recent values to be plugged in the
estimated model is created and stored to disk.

\begin{Shaded}
\begin{Highlighting}[]
\NormalTok{plugin.values<-function(df,max.lag)\{}
        \CommentTok{#  creates the most recent values to be plugged in }
        \CommentTok{# the regression results of lag=1 to max.lag}
        \NormalTok{x.obs=}\KeywordTok{nrow}\NormalTok{(df)}
        \NormalTok{df.lag=}\KeywordTok{lag.hormax}\NormalTok{(df,}\DecValTok{1}\NormalTok{,max.lag}\DecValTok{-1}\NormalTok{)}
        \NormalTok{plugin.values=}\KeywordTok{cbind}\NormalTok{(df[x.obs,],df.lag[x.obs,])}
        \KeywordTok{return}\NormalTok{(plugin.values)}
        \NormalTok{\}}
\CommentTok{# saving function to disk}
\KeywordTok{dump}\NormalTok{(}\StringTok{'plugin.values'}\NormalTok{,}\KeywordTok{paste}\NormalTok{(DirCode,}\StringTok{'/plugin.values.R'}\NormalTok{,}\DataTypeTok{sep=}\StringTok{''}\NormalTok{),}\DataTypeTok{evaluate=}\NormalTok{F)}
\end{Highlighting}
\end{Shaded}

Finally, the BMA package and the three functions just created are
employed to regress y.df on x.df and its lags. The minimum lag is
lag=horizon and the maximum lag is lag=horizon+max.lag. Then the most
recent values are plugged into the model with the highest posterior
probability to compute the h=horizon forecast.

\end{document}
